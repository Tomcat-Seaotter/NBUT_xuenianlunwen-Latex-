 \begin{abstract}
 	% \pagestyle{plain}
        \pagestyle{fancy}
        \songti\zihao{-4}
  	% \thispagestyle{empty}   摘要页的页眉
\par 快递作为新兴行业,正改变人们的消费方式与生活习惯。快递派送工作,是配
送的最后一个环节。目前,快递派送阶段仍然采用以包裹分区承包设立的传统方式,主要存在快递派送区域规划不合理的问题。区域模糊K-Means聚类应用到区域划分中,实现区域规划的密度性合理优化,在快递区域派送中,为实现新的快递派送管理模式,运用算法规划最优派送路径,也能够对提高快递派送效率有较大意义。蚁群算法作为群智能算法的一种,能有效处理组合及聚类等优化问题。\rm{TSP}问题作为组合优化问题的代表,同时K-Means算法作为聚类问题的重要算法,近年来得到了广泛的研究。
\par 当前基于居民区的逐渐扩张,高楼大厦拔地而起。基于高密度的城市化进程中,散点分布的居民区是快递派送的主要目的地。基于对居民区的地理位置分析,希望规划出相对优化的快递站点的设立,便于居民和站点方便寄收快递的同时,为可能设立的快递临时收发点也有一定参考意义,基于此的考虑在区域快递问题上做出了本文的相关研究分析。同时在数据的展示上,制作对居民区的散点分布给读者可视化的图像界面,利用可视化技术将数据于图像融合展示,给读者更好的呈现。
\par 本文在研究居民区分布现状和快递派送现状及相关技术的基础上,对快递派送中存在的问题进行分析,并提出相应的解决规划。对快递派送最优区域划分和路径规划问题进行分析模拟,将其转化成散点图聚类问题和TSP旅行商问题,构建派送最优区域规划和路径规划的数学模型,利用蚁群算法找到相对的最优解,以在最优快递路径规划中的应用模型。
	\\[0.5cm]
	\textbf{关键字}:散点分布;K-Means聚类算法;TSP配送;区域划分;最优解
	\newpage
\end{abstract}