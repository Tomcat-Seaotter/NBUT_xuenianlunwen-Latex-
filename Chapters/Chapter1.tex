\clearpage
\pagenumbering{arabic}
\setcounter{page}{1}
\setcounter{equation}{0}
\section{绪 \quad 论}
\songti\zihao{-4}
\setlength{\baselineskip}{20pt}
\par 2020年,第十一届中国电子商务物流大会提出,”作为重要的基础行业,物流业在保障民生、促进经济社会良好运行、保证全球供应链的稳定方面发挥了重要作用”。当疫情席卷全球时,得益于我国电子商务和快递业的蓬勃发展及相关基础设施的完善,人们可以在避免密切接触的前提下获取所购买的商品,这种购物方式很大程度上保障了基本的安全需求和生活需求,也有利于疫情期间经济的恢复和社会秩序的稳定。国家邮政局数据显示,我国快递行业在2007年至2020年间实现了飞跃式的增长,快递业务量和快递业务收入分别从$2.29$亿件和$520$亿元增长至833.6亿件和8795.4亿元$^{[1]}$。
\par 快递行业近年来受到电子商务发展的影响,进入了快速发展的阶段。国内快递行业也逐渐变得百家争鸣。快递行业的发展同时也推动了互联网经济,促进劳动力就业,快递行业作为日常基础的服务行业日益受到人们的关注。 但目前快递配送“最后一公里”—快递派送,却成为了我国快递发展的瓶颈,经测算,快递在最后一公里的成本占比能达整个快递配送成本的50\%。[5]快递派送的服务质量和效率直接影响物流企业的发展。目前,快递员从营业网点出发执行派送任务,再回到营业网点,其派送路线的选择大多依赖主观经验,对于路线不熟悉缺乏了解,便会造成对派送业务和效率的影响。通过对派送路线的合理规划可以减少派送时间及运营成本,因此目前的人工派送或对未来的无人派送,都需要对派送路线和区域进行合理的规划。\cite{knuthwebsite} 若能利用数学的优势来解决快递派送过程中存在的问题,将有效提高快递企业资源利用率,降低企业运营成本,提高快递派送的效率,进而增强企业市场竞争力。这也将是本课题所要研究和探讨的主要问题,并提出自己的解决方案 \footnote{这是测试文字}。

\subsection{选题背景和研究现状}
\par 随着我国经济贸易持续繁荣发展,中国老字号的邮政的业务服务已经不能完全满足人民的需求,趋于全国物流运营规模的逐渐扩大和交通运输的便利发展,同城配送、异地运输、港澳台及国际快递等业务的逐渐开展,快递已然逐渐发展成为一个新兴的行业。形成了民营与国有企业共同发展竞争的形式。在一方面带来了便利,同时也推动了国家社会的经济发展。由于日前国家地区经济的不断发展,已知城市化的发展促使的交通运输的日益便利、经济日渐繁荣使得人民对物流的需求和要求的提高以及网络贸易迅速普及并急需构建出上线下便捷统一的高效服务,经济的快速发展使得对物流广大需求和更高要求使得快递行业日新月异逐渐成熟。快递业作为基础服务业之一,具有解决地域差异、加快实物传递、信息实时跟踪等多种重要功能,涉及交通运输、基础建设、信息技术、金融投资等多个行业的参与,是邮政业务中不可替代的重要组成部分。伴随着近年来网络贸易的繁荣和电子商务行业转型升级的需要,对快递业的服务能力要求进一步提升。\cite{dirac}
\par 在当前网络贸易经济繁荣茂盛下,大力促进了快递行业的发展并对快递行业有更高的要求。网络购物和网上支付的快速普及,快递业逐渐成为最贴近人们日常生活的产业之一,为大量社会经济活动和个人活动提供了基础的物品运输服务,快递需求日渐增加。快递行业作为经济贸易与交通迅速发展而诞生的新兴产业,优化快递配送为满足居民和顾客的需求和服务是如今的重要研究目标。
如今,快递行业通过近些年的快速发展,在港澳台及国际物流、异地物流、同城派送、生活服务类外卖、跑腿等上有巨大的业务拓展。近年来的迅猛发展使得快递物流配送行业社会话及专业化的趋势,为解决用户对快递的要求以及运营成本的影响增加问题,本文采用聚类规划区域并通过进一步聚类确立可能的分区域点并模拟规划区域配送路线求取最优解。\cite{knuth-fa}


\subsection{研究现状}
\par 快递派送管理,国际上有很多做法,常规的做法是先做零售,当网点布局到一定程度后,再借力网点做快递业务。譬如,在日本,便利店的快递收发功能已非常成熟,人们习惯于到距离家或者公司最近的便利店收发物件,顺便购买一些日用品。由于国情不同,我国在快递派送阶段采取的是基于人工收送货的直接配送模式,国内大多企业对于快递派送的管理依然采用以结果管理过程的传统方式,信息技术的应用较少[3]。 \cite{einstein}
我国快递行业因电子商务而快速发展,近年来发展势头迅猛。高度专业化和社会化是快递行业发展的趋势,营业网点作为快速最末端的服务点,承担着收取快件和派送快件的功能,而派送环节是快递十分重要的一个环节。[5]由于受传统思维的影响,行业内的许多企业的重心依然放在快递物流的运输配送过程中,即跨区域的物流周转,运用大量的先进技术来提高该过程的效率,但随着派送环节发展滞后对用户服务满意度以及运营成本的影响比重增加,行业内开始逐渐提高快递派送管理的重视。
\par 丁浩等研究快递车辆在运输过程中的最短路径选择问题,分别对车辆的最短路径使用遗传算法、A*算法、蚁群算法进行计算并与Dijkstra算法进行对比,结果证明合理的运用Dijkstra算法可以帮助快递企业寻找到最优路径从而降低运输成本;杨从平等针对不同配送路径会影响配送成本的情况,构建基于蚁群算法的快递车辆配送路径优化模型,并通过规则优化后的蚁群算法模型对广西省桂林市的某快递配送网络进行路径优化;宋娟等基于改进遗传算法的同城快递配送模型;倪霖等研究城市快递配送中的路径优化问题,在对多家快递企业共同配送的路径进行模拟分析后,发现快递车辆的装载率对配送效率有较大影响,据此使用改进的遗传算法对装载率与车辆油耗的数学模型进行求解,并对快递同时取送、共同配送、独立配送三种情况的车辆路径进行分别优化。

\subsection{研究的内容和解决问题}
\par 本文将以快递派送中存在的问题为切入点,围绕聚类划分和网络路径规划的区域派送最优路径规划进行展开。对区域聚类划分和派送最优路径问题进行多方研究和分析,将其转化为数学模型,然后运用K-Means聚类算法求解快递区域派送区域进一步规划局部路径。

\subsection{论文的组织结构}
\par 本文共分为五章,各章的内容如下: 
\begin{itemize}
    \item 第一章,绪论。介绍论文研究课题的研究背景和意义,当前存在的问题、课题的相关研究现状以及研究目的。
    \item 第二章,相关技术分析及选择。主要包括对散点数据处理、聚类算法问题的理论分析以及技术选择。
    \item  第三章,模糊聚类算法对散点数据的处理,改善了\rm{K-Means}算法的部分缺点,可以在无基础中心点下进行优化的判断中心点,减少选点带来的误差损失,对数据进行可视化表现为图像。
    \item 第四章,基于MTSP和改进算法TSP的对区域快递派送的划分后的处理,使派件在局部找到最优解问题。
    \item 第五章,总结和展望,对问题的小结以及改进内容。
\end{itemize}