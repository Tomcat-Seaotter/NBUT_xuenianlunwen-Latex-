\clearpage
\setcounter{equation}{0}
\section{总结与展望}
\subsection{总结}
\par 为了进一步提高同城快递配送的时效性,降低配送成本,完成同城快递配送转运中心的选址。为了设计合理的同城配送地铁转运中心布局,结合城市中居民点的散点分布确定的物流需求分布,通过优化的模糊\rm{K-Means}聚类选址模型确定最终的同城配送转运中心。通过对居民区地址的选择统计后,在地图上把居民区可视化表现便于对居民区直观的理解。
\par 基于网络的同城配送转运中心选址研究保证了城市的每个方向区域均设置转运中心,转运中心具有很高的网络节点重要度,并且符合距离顾客需求最近的要求,具有现实意义。本文的研究结果可为我国其他城市地区范围内的网络布局中心确定提供一定参考。其中,对居民区的地址确定服中心一定程度上在对居民服务的方面上(包括但不限于居民老人服务,家政,搬家公司等等问题上获得更大应用)。

\subsection{展望}
\par 本文仅考虑了居民区地址问题,但在居民区数据上由于不能获得较为精准的经纬度表现,对于居民区数据不够精确,所以还需要继续
由于是针对转运中心的路径优化,在实际的生产运营中,存在多个中心之间的影响,并涉及到更大规模的快递配送,即多条线路交叉并行运输,问题规模和复杂程度将进一步扩大,为此后续或可以深入考虑研究关于三个转运中心直至N个转运中心之间的路径优化问题。在设计配送网络时,也已经假设所有的订单信息需求已知且不变,在实际运营中,收件人的需求产生时间不确定和取消退回等问题快递上,若存在需求信息变更的情况,则还不能确定稳定性的问题,因此也还可以继续深入研究收件人取消派件的动态变化的路径优化。 


% \lstinputlisting[language=HTML5]{./codes/example.js}

% \lstinputlisting[language=HTML5]{./codes/example.css}

% \lstinputlisting[language=HTML5]{./codes/example.html}
