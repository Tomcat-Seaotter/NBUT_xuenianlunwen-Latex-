\clearpage
\section{宁波居民区散点分布模型研究同城配送的模式和特性}
由于顾客多是散点状分布,这里涉及分组离散的网络图论问题。在确定派送策略时,我们可以通过合理的假设,将其转化为MTSP问题,因MTSP问题求解难度较大,可把MTSP问题转化为TSP问题解决,即可求出总的最短送货路程。以此最短路程为限制条件,利用基于最小生成树的深度优先搜索算法寻找合适的运行路线,并结合实际情况中的载重限制,对找出的运行路线进行调整修正,即可得出符合要求的运行路线。 
(下面利用TSP的蚁群算法的相关理论对问题进行分析。)

\subsection{MTSP算法模型}
给定$n$个城市($V_1$,$V_2$,…,$V_n$),$m$名旅行商皆以城市$V_0$为起点。令$W_{ij}$表示$V_i$城到$V_j$城的距离。

 $\Phi_{k}=(\nu_{i1},\nu_{i2},\nu_{i3},...,\nu_{in}$ ) 表示一条巡回路线。$\omega  (\Phi_{k})=(W_{i1},W_{i2},W_{i3},...,W_{in}$ 为$\Phi_{k}$巡回的总路程。目标是选择$m$条巡回路线使总路程最小,目标函数如下: 
 $\omega (\Phi _{m} )=\sum_{\Phi _{m}=\Phi _{k}}^{}(W_{i1}+W_{i2}+W_{i3},...+W_{in} )$
当$W_{ij}=W_{ji}$时,称为对称型MTSP。 
目标函数:以总的运行公里数最短为目标 
$min=\sum_{i=1}^{m} \sum_{j=1}^{n}W_{ij}X_{ij} $
(约束条件:最短路径的约束)

\subsection{优化的TSP算法模型求解}
\par 蚁群算法中每只蚂蚁作为一个简单的智能体具有以下特征:蚂蚁按照概率选择下一个要访问的城市,这一概率是要选择城市与当前城市之间距离以及信息素浓度的函数;为了让蚂蚁得到正确的解,已经访问过的城市在解构建完成之前是不能再被访问的,因此每个蚂蚁拥有一个禁忌表来存放已经访问过的城市;当蚂蚁完成解构建之后,会在访问过的边上释放信息素。
$n$个顶点,$m$个旅行商的$MTSP$问题可以转化为$n'=n+m-1$个顶点的TSP问题求解。扩大的$(m-1)$个顶点被称为人造顶点,其距离矩阵$W=\left [ W_{ij}  \right ]_{n\times n}$ 
转化为矩阵
$W'=\left [ W'_{ij}  \right ]_{n\times n}$
,将原问题矩阵:
\begin{equation} 
 W=\left(\begin{array} {cccc}
 0 & w_{01} & \ldots & w_{0 n} \\
 w_{10} & 0 & \ldots & w_{1 n} \\
 \cdots  & \cdots & \cdots  & \cdots  \\
 w_{m 0} & w_{m 1} & \cdots  & 0
 \end{array}
 \right)
 \end{equation} 
转化后建立的TSP模型如下:
目标函数:
\begin{equation} 
 min=\sum_{i=1}^{m}\sum_{j=1}^{n}W'_{0}X_{0}    
 \end{equation}

约束条件:
\begin{equation}
 \left\{\begin{array}{l}
\sum_{i=1}^{m} x_{i j}=1, j=1,2, \ldots, n \\
\sum_{j=1}^{n} x_{i j}=1, i=1,2, \ldots, m \\
n_{i}-n_{j}+n X_{j} \leq m-1, i \neq j, i, j=2,3, \ldots, n \\
n_{j} \geq 0, j=1,2, \ldots, n
\end{array}\right.
\end{equation}


\subsection{快递区域配送应用}
\par 某快递公司,假定所有快件在早上7点钟到达,早上9 点钟开始派送,要求与当天17点之前必须派送完毕,每个业务员每天平均工作时间不超过6小时,在每个送货点停留的时间为10分钟,途中速度为25km/h,每次出发最多能带25千克的重量。为了计算方便,将快件一律用重量来衡量,平均每天收到总重量为固定种类,并且假设街道平行于坐标轴方向。
那么需要计算出的合理派送策略需要作如下假设:
\begin{enumerate}
    \item 公司总部每次发货的对象是无差别的;
    \item 尽量满足每条路线负重总量均衡; 
    \item 业务员行走拐弯的时间,路上的意外事故的耽搁时间忽略; 
    \item 街道平行于坐标轴方向。 
\end{enumerate}
\par 显然这是一个多旅行商问题,可将其转化为单旅行商问题,而对于TSP问题利用LINGO软件利用蚁群算法求解,解得最短总路程为$492$公里。

\begin{table}[!ht]
    \centering
        \caption{线路运行情况}
    \begin{tabular}{|c|c|c|c|c|}
    \hline 
  路线编号   &  路线        &路程(公里) & 负重     &	时间(小时)  \\ \hline
       线1  &  0-16-17-28-29-0 &	90	& 23.4    &	4.27  \\ \hline
       线2  & 0-2-5-15-18-0    &	56	& 24.8    &	  2.91   \\ \hline
       线3  & 0-4-24-25-0      &	68	& 22.7    &	 3.22   \\ \hline
       线4  & 0-6-7-14-26-0    &   74  &  21     &	3.63   \\ \hline
       线5  & 0-19-30-23-21-0  &	92	& 20.6   &	4.35   \\ \hline
       线6  & 0-3-13-27-20-0   &	68	& 27.2   &	3.39   \\ \hline
       线7  & 0-8-12-11-0      &	46	& 19.1   &	2.34   \\ \hline
       线8  & 0-1-9-10-22-0    &	46	& 22.7   &	2.51   \\ \hline
    \end{tabular}
\end{table}

\par 根据工作时间均衡的原则,分别将路线 2 和路线8 合并,路线 6 和路线 7 合并,合并后每条路线由一个业务员送货,因此只需要6个业务员。最终调整后的每个业务员的运行路线如表 3 和图 2 所示,业务员的总运行路程为 546 公里。由于派送过程中有负重的约束条件,使得每个业务员的工作时间较短,多在4 个小时左右,相较于每个业务员最长工作时间6小时,表面上看利用率不足。由此快递局部在散点下派送的通过优化模型的方法,继而得到近似解。对配送员的配送路径进行优化,建立合理的模型假设、设定模型参数,考虑配送任务中取货点和送货点配对有序,将取货路径和送货路径结合起来,建立同城配送取送货路径的联合优化模型,使用TSP遗传算法求解,得到总配送成本最小的最佳配送路径。